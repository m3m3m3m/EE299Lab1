In this section, we will show all of our Arduino code that were developed and used in this lab.
\lstset{language=C}
\begin{lstlisting}[caption = blink1.ino]
// Pin 13
int led = 13;

// the setup routine runs once when you press reset:
void setup() 
{                
	// initialize the digital pin as an output.
	pinMode(led, OUTPUT);     
}

// the loop routine runs over and over again forever:
void loop() 
{
	digitalWrite(led, HIGH);   // turn the LED on (HIGH is the voltage level)
	delay(2000);               // wait for two seconds
	digitalWrite(led, LOW);    // turn the LED off by making the voltage LOW
	delay(1000);               // wait for a second
}
\end{lstlisting}
\begin{lstlisting}[caption = blink2.ino]
// Pin 13 
int led = 13;

// the setup routine runs once when you press reset:
void setup() 
{                
	// initialize the digital pin as an output.
	pinMode(led, OUTPUT);     
}

// the loop routine runs over and over again forever:
void loop() 
{
	digitalWrite(led, HIGH);   // turn the LED on (HIGH is the voltage level)
	delay(1000);               // wait for a second
	digitalWrite(led, LOW);    // turn the LED off by making the voltage LOW
	delay(1000);               // wait for a second
	digitalWrite(led, HIGH);   // turn the LED on (HIGH is the voltage level)
	delay(1000);               // wait for a second
	digitalWrite(led, LOW);    // turn the LED off by making the voltage LOW
	delay(1000);               // wait for a second
	
	digitalWrite(led, HIGH);   // turn the LED on (HIGH is the voltage level)
	delay(2000);               // wait for two seconds
	digitalWrite(led, LOW);    // turn the LED off by making the voltage LOW
	delay(1000);               // wait for a second
	digitalWrite(led, HIGH);   // turn the LED on (HIGH is the voltage level)
	delay(2000);               // wait for two seconds
	digitalWrite(led, LOW);    // turn the LED off by making the voltage LOW
	delay(1000);               // wait for a second
}
\end{lstlisting}
\begin{lstlisting}[caption = blink3.ino]
// LED Pin 8
int led = 8;

// Button Pin 9
int button = 9;

// the setup routine runs once when you press reset:
void setup() 
{
	// initialize the digital pin as an output.
	pinMode(led, OUTPUT);     
	// 
	pinMode(button, INPUT);
}

// the loop routine runs over and over again forever:
void loop() 
{
	if (digitalRead(button)) 
	{
		digitalWrite(led, HIGH);   // turn the LED on (HIGH is the voltage level)
	} 
	else 
	{
		digitalWrite(led, LOW);   // turn the LED off
	}
}
\end{lstlisting}
\begin{lstlisting}[caption = LCD1.ino]
#include <LiquidCrystal.h>

LiquidCrystal lcd(2, 3, 4, 5, 6, 7, 8); // Bus 1

void setup() 
{
	lcd.begin(16, 2); // 16 * 2 character
	
	lcd.print("Man Muhan");
	lcd.setCursor(0, 1);
	lcd.print("Mingxiao");
}

void loop() 
{
	// do nothing
}
\end{lstlisting}
\begin{lstlisting}[caption = LCD2.ino]
#include <LiquidCrystal.h>

LiquidCrystal lcd(2, 3, 4, 5, 6, 7, 8); // port 1
int buttonLeft = 10;
int buttonRight = 9;
int col = 0;
int row = 0;


void moveLeft() 
{
	lcd.clear();
	if (col > 1) col -= 1;
	else if (row > 0) 
	{
		row -= 1;
		col = 16;
	}
	lcd.setCursor(col, row);
	lcd.print("X");
}

void moveRight() 
{
	lcd.clear();
	if (col < 16)
	col += 1;
	else 
	{
		row += 1;
		col = 1;
	}
	lcd.setCursor(col, row);
	lcd.print("X");
}

void setup() 
{
	pinMode(buttonLeft, INPUT);
	pinMode(buttonRight, INPUT);
	lcd.begin(16, 2); // 16 * 2 character
	lcd.autoscroll();
	//  lcd.cursor();
	moveRight();
}

void loop() 
{
	if (digitalRead(buttonLeft)) 
	{
		moveLeft();
		delay(300);
	}
	if (digitalRead(buttonRight)) 
	{
		moveRight();
		delay(300);
	}
}
\end{lstlisting}
\begin{lstlisting}[caption = 12i2cm.ino]
#include <Wire.h>

void setup()
{
	Wire.begin();         // join i2c bus (address optional for master)
	Serial.begin(9600);       // start serial for output
}

byte lightOn = 1;

void loop()
{
	Wire.beginTransmission(4);  // transmit to device #4
	Wire.write(lightOn);      // send the signal of whether the LED is on
	
	Wire.endTransmission();     // stop transmitting
	if(lightOn == 1)
	{
		Serial.print("LED on\n");
		lightOn = 0;
	}
	else 
	{
		Serial.print("LED off\n");
		lightOn = 1;
	}
	delay(500);
}
\end{lstlisting}
\begin{lstlisting}[caption = 12i2cs.ino]
#include <Wire.h>
// Pin 13
int led = 13;
void setup()
{
	Wire.begin(4);         // join i2c bus with address #4
	Wire.onReceive(receiveEvent); // register event
	Serial.begin(9600);       // start serial for output
	pinMode(led,OUTPUT);
}
void loop()
{
	delay(100);
}
// function that executes whenever data is received from master
// this function is registered as an event, see setup()
void receiveEvent(int howMany)
{
	while(1 <Wire.available())  // loop through all but the last
	{
		Serial.print("received");
		bool on = Wire.read();    // receive byte as a character
		if(on)
		{
			digitalWrite(led,HIGH);
			Serial.print("LED on\n");
		}
		else
		{
			digitalWrite(led,LOW);
			Serial.print("LED off\n");
		}
	}
}
\end{lstlisting}